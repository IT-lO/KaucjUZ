\documentclass[a4paper,11pt]{article}
\usepackage[utf8]{inputenc}
\usepackage[polish]{babel}
\usepackage[T1]{fontenc}
\usepackage{geometry}
\usepackage{longtable}
\usepackage{array}
\usepackage{booktabs}
\usepackage{xcolor}
\usepackage{enumitem}

\geometry{
    left=2cm,
    right=2cm,
    top=2cm,
    bottom=2cm
}

\title{Scenariusze przypadków użycia\\Automat do butelek}
\author{}
\date{}

\begin{document}

\maketitle

\section{UC1: Wprowadzenie butelki}

\begin{longtable}{|p{4cm}|p{11cm}|}
\hline
\textbf{Element} & \textbf{Opis} \\
\hline
\endfirsthead

\hline
\textbf{Element} & \textbf{Opis} \\
\hline
\endhead

\hline
\endfoot

\textbf{Przypadek użycia} & Wprowadzenie butelki do automatu \\
\hline

\textbf{Aktor} & Użytkownik \\
\hline

\textbf{Opis} & Użytkownik wkłada butelkę do automatu w celu jej recyklingu i otrzymania zwrotu pieniędzy lub darmowej utylizacji \\
\hline

\textbf{Warunki wstępne} & 
\begin{itemize}[leftmargin=*, nosep]
    \item Automat jest włączony i sprawny
    \item Automat ma miejsce w kontenerze
    \item Użytkownik posiada butelkę do zwrotu
\end{itemize} \\
\hline

\textbf{Główny scenariusz} & 
\begin{enumerate}[leftmargin=*, nosep]
    \item Użytkownik umieszcza butelkę w otworze przyjmującym
    \item System zczytuje kod kreskowy butelki
    \item System określa materiał butelki (plastik/szkło/aluminium)
    \item System określa pojemność butelki
    \item System określa poziom zgniecenia butelki
    \item System akceptuje butelkę
    \item System sortuje butelkę po materiale
    \item System zgniata butelkę (jeśli plastikowa)
    \item System przemieszcza butelkę do odpowiedniego kontenera
    \item System aktualizuje wysokość uznania na wyświetlaczu
    \item System aktualizuje liczbę przyjętych butelek
    \item System wyświetla timer przed ponownym wrzuceniem
\end{enumerate} \\
\hline

\textbf{Alternatywne scenariusze} & 
\textbf{A1: Butelka bez kodu kreskowego}
\begin{itemize}[leftmargin=*, nosep]
    \item 2a. System nie może odczytać kodu kreskowego
    \item 2b. System odrzuca butelkę
    \item 2c. System wyświetla komunikat o błędzie
\end{itemize}

\vspace{0.3cm}
\textbf{A2: Butelka zbyt zgnieciona}
\begin{itemize}[leftmargin=*, nosep]
    \item 5a. System stwierdza zbyt duży poziom zgniecenia
    \item 5b. System oferuje darmową utylizację
    \item 5c. Użytkownik potwierdza lub odrzuca
    \item 5d. System przyjmuje bez zwrotu lub odrzuca butelkę
\end{itemize}

\vspace{0.3cm}
\textbf{A3: Butelka niekwalifikująca się}
\begin{itemize}[leftmargin=*, nosep]
    \item 5a. System stwierdza, że butelka nie spełnia kryteriów
    \item 5b. System odrzuca butelkę
    \item 5c. System wyświetla przyczynę odrzucenia
\end{itemize}

\vspace{0.3cm}
\textbf{A4: Kontener pełny}
\begin{itemize}[leftmargin=*, nosep]
    \item 9a. System wykrywa brak miejsca w kontenerze
    \item 9b. System przerywa przyjmowanie butelek
    \item 9c. System wyświetla komunikat "Automat pełny"
    \item 9d. System wysyła powiadomienie do serwisanta
\end{itemize} \\
\hline

\textbf{Warunki końcowe} & 
\begin{itemize}[leftmargin=*, nosep]
    \item Butelka została przyjęta i umieszczona w odpowiednim kontenerze LUB została odrzucona
    \item Wysokość uznania została zaktualizowana (jeśli przyjęto)
    \item Liczba butelek została zaktualizowana (jeśli przyjęto)
    \item System gotowy do przyjęcia kolejnej butelki
\end{itemize} \\
\hline

\end{longtable}

\newpage

\section{UC6: Wypłata środków}

\begin{longtable}{|p{4cm}|p{11cm}|}
\hline
\textbf{Element} & \textbf{Opis} \\
\hline
\endfirsthead

\hline
\textbf{Element} & \textbf{Opis} \\
\hline
\endhead

\hline
\endfoot

\textbf{Przypadek użycia} & Wypłata środków za butelki \\
\hline

\textbf{Aktor} & Użytkownik \\
\hline

\textbf{Opis} & Użytkownik kończy wkładanie butelek i wybiera sposób wypłaty zgromadzonych środków \\
\hline

\textbf{Warunki wstępne} & 
\begin{itemize}[leftmargin=*, nosep]
    \item Użytkownik wrzucił przynajmniej jedną butelkę
    \item Wysokość uznania jest większa od zera
    \item Automat jest sprawny
\end{itemize} \\
\hline

\textbf{Główny scenariusz} & 
\begin{enumerate}[leftmargin=*, nosep]
    \item Użytkownik wybiera "Zakończ transakcję" na wyświetlaczu
    \item System wyświetla wysokość zgromadzonych środków
    \item System wyświetla dostępne metody wypłaty
    \item Użytkownik wybiera metodę wypłaty (np. karta sklepu)
    \item System przetwarza wypłatę
    \item System potwierdza wypłatę
    \item System wyświetla podsumowanie transakcji
    \item System wysyła powiadomienie o transakcji do systemu sklepu
    \item System resetuje licznik i wraca do stanu początkowego
\end{enumerate} \\
\hline

\textbf{Alternatywne scenariusze} & 
\textbf{A1: Wypłata na kartę sklepu}
\begin{itemize}[leftmargin=*, nosep]
    \item 4a. Użytkownik wybiera "Karta sklepu"
    \item 4b. Użytkownik skanuje kartę sklepu
    \item 4c. System dodaje środki do karty
\end{itemize}

\vspace{0.3cm}
\textbf{A2: Wypłata w monetach}
\begin{itemize}[leftmargin=*, nosep]
    \item 4a. Użytkownik wybiera "Monety"
    \item 4b. System wydaje monety
    \item 4c. System wyświetla "Odbierz monety"
\end{itemize}

\vspace{0.3cm}
\textbf{A3: Wypłata przez aplikację}
\begin{itemize}[leftmargin=*, nosep]
    \item 4a. Użytkownik wybiera "Aplikacja sklepu"
    \item 4b. Użytkownik skanuje kod QR aplikacją
    \item 4c. System przesyła środki do aplikacji
\end{itemize}

\vspace{0.3cm}
\textbf{A4: Wypłata bonem ze zniżką}
\begin{itemize}[leftmargin=*, nosep]
    \item 4a. Użytkownik wybiera "Bon ze zniżką"
    \item 4b. System drukuje bon z kodem kreskowym
    \item 4c. System wyświetla "Odbierz bon"
\end{itemize}

\vspace{0.3cm}
\textbf{A5: Wypłata na kartę płatniczą}
\begin{itemize}[leftmargin=*, nosep]
    \item 4a. Użytkownik wybiera "Karta"
    \item 4b. Użytkownik zbliża kartę do terminala
    \item 4c. System przetwarza zwrot na kartę
\end{itemize}

\vspace{0.3cm}
\textbf{A6: Brak pieniędzy w automacie}
\begin{itemize}[leftmargin=*, nosep]
    \item 5a. System wykrywa brak monet
    \item 5b. System ukrywa opcję "Monety"
    \item 5c. System oferuje alternatywne metody
\end{itemize}

\vspace{0.3cm}
\textbf{A7: Brak papieru do bonów}
\begin{itemize}[leftmargin=*, nosep]
    \item 5a. System wykrywa brak papieru
    \item 5b. System ukrywa opcję "Bon ze zniżką"
    \item 5c. System oferuje alternatywne metody
\end{itemize}

\vspace{0.3cm}
\textbf{A8: Anulowanie transakcji}
\begin{itemize}[leftmargin=*, nosep]
    \item 4a. Użytkownik wybiera "Anuluj"
    \item 4b. System anuluje transakcję
    \item 4c. System wraca do przyjmowania butelek
\end{itemize} \\
\hline

\textbf{Warunki końcowe} & 
\begin{itemize}[leftmargin=*, nosep]
    \item Środki zostały wypłacone wybraną metodą
    \item Powiadomienie o transakcji zostało wysłane
    \item System został zresetowany do stanu początkowego
    \item Wyświetlacz pokazuje ekran powitalny
\end{itemize} \\
\hline

\end{longtable}

\newpage

\section{UC10: Maintenance automatu}

\begin{longtable}{|p{4cm}|p{11cm}|}
\hline
\textbf{Element} & \textbf{Opis} \\
\hline
\endfirsthead

\hline
\textbf{Element} & \textbf{Opis} \\
\hline
\endhead

\hline
\endfoot

\textbf{Przypadek użycia} & Konserwacja i serwis automatu \\
\hline

\textbf{Aktor} & Serwisant \\
\hline

\textbf{Opis} & Serwisant wykonuje czynności konserwacyjne automatu w celu utrzymania jego sprawności \\
\hline

\textbf{Warunki wstępne} & 
\begin{itemize}[leftmargin=*, nosep]
    \item Serwisant ma dostęp do automatu
    \item Serwisant posiada uprawnienia i klucze
    \item System wysłał powiadomienie o konieczności serwisu LUB jest zaplanowana konserwacja
\end{itemize} \\
\hline

\textbf{Główny scenariusz} & 
\begin{enumerate}[leftmargin=*, nosep]
    \item Serwisant loguje się do trybu serwisowego
    \item System wyświetla status automatu
    \item Serwisant sprawdza poziom napełnienia kontenerów
    \item Serwisant opróżnia kontenery z butelkami
    \item Serwisant sprawdza stan rolki papieru
    \item Serwisant wymienia rolkę papieru (jeśli potrzeba)
    \item Serwisant sprawdza stan pieniędzy
    \item Serwisant uzupełnia monety (jeśli potrzeba)
    \item Serwisant sprawdza stan techniczny elementów
    \item Serwisant zatwierdza zakończenie konserwacji
    \item System wysyła powiadomienie o zakończeniu serwisu
    \item System wraca do trybu normalnego
\end{enumerate} \\
\hline

\textbf{Alternatywne scenariusze} & 
\textbf{A1: Poważna awaria}
\begin{itemize}[leftmargin=*, nosep]
    \item 9a. Serwisant wykrywa poważną usterkę
    \item 9b. Serwisant przełącza automat w tryb "Out of commission"
    \item 9c. System wyświetla "Automat nieczynny"
    \item 9d. System blokuje przyjmowanie butelek
    \item 9e. System wysyła powiadomienie o awarii
\end{itemize}

\vspace{0.3cm}
\textbf{A2: Kontenery prawie pełne}
\begin{itemize}[leftmargin=*, nosep]
    \item 3a. System wykrywa napełnienie powyżej 80\%
    \item 3b. System wysyła powiadomienie o zbliżającym się przepełnieniu
\end{itemize}

\vspace{0.3cm}
\textbf{A3: Papier kończy się}
\begin{itemize}[leftmargin=*, nosep]
    \item 5a. System wykrywa, że papier kończy się
    \item 5b. System wysyła powiadomienie o konieczności wymiany
\end{itemize}

\vspace{0.3cm}
\textbf{A4: Pieniądze kończą się}
\begin{itemize}[leftmargin=*, nosep]
    \item 7a. System wykrywa, że monet jest mniej niż 20\%
    \item 7b. System wysyła powiadomienie o konieczności uzupełnienia
\end{itemize}

\vspace{0.3cm}
\textbf{A5: Konserwacja prewencyjna}
\begin{itemize}[leftmargin=*, nosep]
    \item 1a. Serwisant rozpoczyna zaplanowaną konserwację
    \item 1b. System nie wysyłał wcześniej powiadomień
    \item 1c. Serwisant wykonuje standardowe czynności kontrolne
\end{itemize} \\
\hline

\textbf{Warunki końcowe} & 
\begin{itemize}[leftmargin=*, nosep]
    \item Kontenery zostały opróżnione
    \item Papier został uzupełniony (jeśli potrzeba)
    \item Monety zostały uzupełnione (jeśli potrzeba)
    \item System jest w trybie normalnym LUB w trybie "Out of commission"
    \item Powiadomienie o stanie po serwisie zostało wysłane
\end{itemize} \\
\hline

\end{longtable}

\newpage

\section{UC11: Wysyłanie powiadomień}

\begin{longtable}{|p{4cm}|p{11cm}|}
\hline
\textbf{Element} & \textbf{Opis} \\
\hline
\endfirsthead

\hline
\textbf{Element} & \textbf{Opis} \\
\hline
\endhead

\hline
\endfoot

\textbf{Przypadek użycia} & Wysyłanie powiadomień do systemu sklepu \\
\hline

\textbf{Aktor} & System sklepu/Aplikacja \\
\hline

\textbf{Opis} & Automat automatycznie wysyła powiadomienia o stanie technicznym i transakcjach do centralnego systemu \\
\hline

\textbf{Warunki wstępne} & 
\begin{itemize}[leftmargin=*, nosep]
    \item Automat ma połączenie z systemem sklepu
    \item Wystąpiło zdarzenie wymagające powiadomienia
\end{itemize} \\
\hline

\textbf{Główny scenariusz} & 
\begin{enumerate}[leftmargin=*, nosep]
    \item System wykrywa zdarzenie wymagające powiadomienia
    \item System przygotowuje dane powiadomienia
    \item System wysyła powiadomienie POST do systemu sklepu
    \item System otrzymuje potwierdzenie odbioru
    \item System loguje wysłane powiadomienie
\end{enumerate} \\
\hline

\textbf{Alternatywne scenariusze} & 
\textbf{A1: Powiadomienie o stanie technicznym - papier}
\begin{itemize}[leftmargin=*, nosep]
    \item 1a. System wykrywa, że papier kończy się (<20\%)
    \item 1b. System wysyła powiadomienie "LOW\_PAPER"
\end{itemize}

\vspace{0.3cm}
\textbf{A2: Powiadomienie o stanie technicznym - kontener}
\begin{itemize}[leftmargin=*, nosep]
    \item 1a. System wykrywa, że miejsce w kontenerze się kończy (>80\%)
    \item 1b. System wysyła powiadomienie "CONTAINER\_FULL"
\end{itemize}

\vspace{0.3cm}
\textbf{A3: Powiadomienie o stanie technicznym - pieniądze}
\begin{itemize}[leftmargin=*, nosep]
    \item 1a. System wykrywa, że pieniądze się kończą (<20\%)
    \item 1b. System wysyła powiadomienie "LOW\_COINS"
\end{itemize}

\vspace{0.3cm}
\textbf{A4: Powiadomienie o transakcji}
\begin{itemize}[leftmargin=*, nosep]
    \item 1a. Użytkownik zakończył transakcję
    \item 1b. System wysyła powiadomienie "TRANSACTION\_COMPLETE" z danymi transakcji
\end{itemize}

\vspace{0.3cm}
\textbf{A5: Brak połączenia}
\begin{itemize}[leftmargin=*, nosep]
    \item 3a. System nie może połączyć się z serwerem
    \item 3b. System zapisuje powiadomienie w kolejce
    \item 3c. System próbuje wysłać ponownie co 5 minut
\end{itemize}

\vspace{0.3cm}
\textbf{A6: Krytyczna awaria}
\begin{itemize}[leftmargin=*, nosep]
    \item 1a. System wykrywa krytyczną awarię
    \item 1b. System wysyła powiadomienie "CRITICAL\_ERROR" z najwyższym priorytetem
\end{itemize} \\
\hline

\textbf{Warunki końcowe} & 
\begin{itemize}[leftmargin=*, nosep]
    \item Powiadomienie zostało wysłane i potwierdzone LUB zapisane w kolejce
    \item Zdarzenie zostało zalogowane w systemie
    \item System sklepu otrzymał informację o stanie automatu
\end{itemize} \\
\hline

\end{longtable}

\newpage

\section{UC12: Obsługa wyświetlacza/interfejsu}

\begin{longtable}{|p{4cm}|p{11cm}|}
\hline
\textbf{Element} & \textbf{Opis} \\
\hline
\endfirsthead

\hline
\textbf{Element} & \textbf{Opis} \\
\hline
\endhead

\hline
\endfoot

\textbf{Przypadek użycia} & Interakcja użytkownika z wyświetlaczem automatu \\
\hline

\textbf{Aktor} & Użytkownik \\
\hline

\textbf{Opis} & Użytkownik nawiguje po interfejsie automatu, wybiera opcje i otrzymuje informacje zwrotne \\
\hline

\textbf{Warunki wstępne} & 
\begin{itemize}[leftmargin=*, nosep]
    \item Automat jest włączony
    \item Wyświetlacz działa poprawnie
\end{itemize} \\
\hline

\textbf{Główny scenariusz} & 
\begin{enumerate}[leftmargin=*, nosep]
    \item System wyświetla ekran powitalny
    \item Użytkownik wybiera język z dostępnych opcji
    \item System zmienia język interfejsu
    \item System wyświetla instrukcję wkładania butelek
    \item System wyświetla aktualną wysokość uznania (0,00 zł)
    \item System wyświetla liczbę przyjętych butelek (0 szt.)
    \item Użytkownik wkłada butelki (UC1)
    \item System aktualizuje wysokość uznania po każdej butelce
    \item System aktualizuje liczbę butelek
    \item System wyświetla timer przed ponownym wrzuceniem (3 sek.)
    \item Po upływie timera system sygnalizuje gotowość
    \item Użytkownik wybiera "Zakończ transakcję"
    \item System wyświetla ekran wyboru metody wypłaty
    \item Użytkownik wybiera metodę wypłaty
    \item System przetwarza wypłatę (UC6)
    \item System wyświetla podsumowanie transakcji
    \item System wyświetla podziękowanie
    \item System wraca do ekranu powitalnego
\end{enumerate} \\
\hline

\textbf{Alternatywne scenariusze} & 
\textbf{A1: Zmiana języka w trakcie}
\begin{itemize}[leftmargin=*, nosep]
    \item 7a. Użytkownik wybiera zmianę języka w menu
    \item 7b. System wyświetla dostępne języki
    \item 7c. System zmienia język bez przerywania transakcji
\end{itemize}

\vspace{0.3cm}
\textbf{A2: Timeout bezczynności}
\begin{itemize}[leftmargin=*, nosep]
    \item 7a. Użytkownik nie wkłada butelki przez 60 sekund
    \item 7b. System wyświetla komunikat "Czy chcesz kontynuować?"
    \item 7c. Jeśli brak reakcji przez 30 sek., system przechodzi do wypłaty
\end{itemize}

\vspace{0.3cm}
\textbf{A3: Anulowanie przed wypłatą}
\begin{itemize}[leftmargin=*, nosep]
    \item 12a. Użytkownik wybiera "Anuluj" lub "Wrzuć więcej butelek"
    \item 12b. System wraca do ekranu przyjmowania butelek
\end{itemize}

\vspace{0.3cm}
\textbf{A4: Błąd przyjęcia butelki}
\begin{itemize}[leftmargin=*, nosep]
    \item 7a. System odrzuca butelkę
    \item 7b. System wyświetla komunikat o przyczynie odrzucenia
    \item 7c. System pokazuje instrukcję poprawnego wkładania
    \item 7d. System wraca do przyjmowania po 5 sekundach
\end{itemize}

\vspace{0.3cm}
\textbf{A5: Automat pełny}
\begin{itemize}[leftmargin=*, nosep]
    \item 1a. System wykrywa pełny kontener
    \item 1b. System wyświetla "Automat tymczasowo nieczynny"
    \item 1c. System blokuje możliwość rozpoczęcia transakcji
\end{itemize}

\vspace{0.3cm}
\textbf{A6: Tryb serwisowy}
\begin{itemize}[leftmargin=*, nosep]
    \item 1a. Serwisant aktywuje tryb serwisowy
    \item 1b. System wyświetla interfejs serwisowy
    \item 1c. System blokuje dostęp dla zwykłych użytkowników
\end{itemize} \\
\hline

\textbf{Warunki końcowe} & 
\begin{itemize}[leftmargin=*, nosep]
    \item Użytkownik zakończył interakcję z automatem
    \item System wrócił do ekranu powitalnego
    \item Wszystkie dane transakcji zostały przetworzone
    \item Interfejs jest gotowy dla kolejnego użytkownika
\end{itemize} \\
\hline

\end{longtable}

\newpage

\section{UC9: Przechowywanie butelek}

\begin{longtable}{|p{4cm}|p{11cm}|}
\hline
\textbf{Element} & \textbf{Opis} \\
\hline
\endfirsthead

\hline
\textbf{Element} & \textbf{Opis} \\
\hline
\endhead

\hline
\endfoot

\textbf{Przypadek użycia} & Przetwarzanie i przechowywanie przyjętych butelek \\
\hline

\textbf{Aktor} & System (proces automatyczny) \\
\hline

\textbf{Opis} & Automat automatycznie sortuje, przetwarza i przechowuje przyjęte butelki w odpowiednich kontenerach \\
\hline

\textbf{Warunki wstępne} & 
\begin{itemize}[leftmargin=*, nosep]
    \item Butelka została przyjęta i zweryfikowana
    \item Materiał butelki został określony
    \item W kontenerach jest miejsce
\end{itemize} \\
\hline

\textbf{Główny scenariusz} & 
\begin{enumerate}[leftmargin=*, nosep]
    \item System identyfikuje materiał butelki (plastik/szkło/aluminium)
    \item System kieruje butelkę na odpowiedni tor sortujący
    \item Jeśli butelka plastikowa: system aktywuje mechanizm zgniatający
    \item System zgniata butelkę do 1/3 objętości
    \item System przemieszcza butelkę do odpowiedniego kontenera
    \item System inkrementuje licznik butelek w kontenerze
    \item System sprawdza poziom napełnienia kontenera
    \item Jeśli kontener >80\%: system wysyła powiadomienie
    \item System potwierdza umieszczenie butelki
\end{enumerate} \\
\hline

\textbf{Alternatywne scenariusze} & 
\textbf{A1: Butelka szklana}
\begin{itemize}[leftmargin=*, nosep]
    \item 3a. System wykrywa materiał: szkło
    \item 3b. System pomija zgniatanie
    \item 3c. System ostrożnie przemieszcza do kontenera szkła
\end{itemize}

\vspace{0.3cm}
\textbf{A2: Butelka aluminiowa}
\begin{itemize}[leftmargin=*, nosep]
    \item 3a. System wykrywa materiał: aluminium
    \item 3b. System pomija zgniatanie (lub lekko zgniata)
    \item 3c. System przemieszcza do kontenera aluminium
\end{itemize}

\vspace{0.3cm}
\textbf{A3: Kontener prawie pełny}
\begin{itemize}[leftmargin=*, nosep]
    \item 7a. System wykrywa napełnienie >80\%
    \item 7b. System wysyła powiadomienie "CONTAINER\_ALMOST\_FULL"
    \item 7c. System kontynuuje przyjmowanie do 95\%
\end{itemize}

\vspace{0.3cm}
\textbf{A4: Kontener pełny}
\begin{itemize}[leftmargin=*, nosep]
    \item 7a. System wykrywa napełnienie >95\%
    \item 7b. System wysyła powiadomienie "CONTAINER\_FULL"
    \item 7c. System blokuje przyjmowanie kolejnych butelek tego typu
    \item 7d. System wyświetla komunikat użytkownikowi
\end{itemize}

\vspace{0.3cm}
\textbf{A5: Błąd mechanizmu zgniatającego}
\begin{itemize}[leftmargin=*, nosep]
    \item 4a. System wykrywa usterkę mechanizmu
    \item 4b. System pomija zgniatanie
    \item 4c. System umieszcza butelkę niezgniecioną
    \item 4d. System wysyła powiadomienie o usterce
\end{itemize}

\vspace{0.3cm}
\textbf{A6: Zablokowanie toru}
\begin{itemize}[leftmargin=*, nosep]
    \item 5a. System wykrywa blokadę na torze transportowym
    \item 5b. System próbuje usunąć blokadę (odwrócenie silnika)
    \item 5c. Jeśli niepowodzenie: system przechodzi w tryb awaryjny
    \item 5d. System wysyła powiadomienie "TRANSPORT\_BLOCKED"
\end{itemize} \\
\hline

\textbf{Warunki końcowe} & 
\begin{itemize}[leftmargin=*, nosep]
    \item Butelka została umieszczona w odpowiednim kontenerze
    \item Liczniki kontenerów zostały zaktualizowane
    \item System monitoruje poziom napełnienia
    \item Powiadomienia zostały wysłane (jeśli potrzeba)
\end{itemize} \\
\hline

\end{longtable}

\end{document}